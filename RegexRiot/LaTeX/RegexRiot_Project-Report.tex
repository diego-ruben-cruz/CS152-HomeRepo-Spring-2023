\documentclass[conference]{IEEEtran}
% \IEEEoverridecommandlockouts
% The preceding line is only needed to identify funding in the first footnote. If that is unneeded, please comment it out.
\usepackage{cite}
\usepackage{amsmath,amssymb,amsfonts}
\usepackage{algorithmic}
\usepackage{graphicx}
\graphicspath{{./images/}}
\usepackage{textcomp}
\usepackage{xcolor}
\usepackage{listings}
\lstdefinelanguage{JavaScript}{
  keywords={const, let, var, function, return, if, else, for, while, do, switch, case, break},
  keywordstyle=\color{blue}\bfseries,
  ndkeywords={class, export, boolean, throw, implements, import, this},
  ndkeywordstyle=\color{darkgray}\bfseries,
  identifierstyle=\color{black},
  sensitive=false,
  comment=[l]{//},
  morecomment=[s]{/*}{*/},
  commentstyle=\color{purple}\ttfamily,
  stringstyle=\color{red}\ttfamily,
  morestring=[b]',
  morestring=[b]"
}
\lstdefinelanguage{Java}{
  keywords={abstract, assert, boolean, break, byte, case, catch, char, class, const, continue, default, do, double, else, enum, extends, final, finally, float, for, goto, if, implements, import, instanceof, int, interface, long, native, new, package, private, protected, public, return, short, static, strictfp, super, switch, synchronized, this, throw, throws, transient, try, void, volatile, while},
  keywordstyle=\color{blue}\bfseries,
  ndkeywords={boolean, byte, char, class, double, enum, float, int, long, short, var},
  ndkeywordstyle=\color{purple}\bfseries,
  identifierstyle=\color{black},
  sensitive=false,
  comment=[l]{//},
  morecomment=[s]{/*}{*/},
  commentstyle=\color{gray}\ttfamily,
  stringstyle=\color{red}\ttfamily,
  morestring=[b]',
  morestring=[b]"
}

\def\BibTeX{{\rm B\kern-.05em{\sc i\kern-.025em b}\kern-.08em
    T\kern-.1667em\lower.7ex\hbox{E}\kern-.125emX}}
\begin{document}

\title{RegexRiot \\A Java Library for Building Regular Expressions}\\

\author{\IEEEauthorblockN{Diego Cruz}
    \IEEEauthorblockA{\textit{Dept. of Computer Science} \\
        \textit{San Jose State University}\\
        San Jose, United States \\
        diego.cruz@sjsu.edu}
    \and
    \IEEEauthorblockN{Rahul Kandekar}
    \IEEEauthorblockA{\textit{Dept. of Computer Science} \\
        \textit{San Jose State University}\\
        San Jose, United States \\
        rahul.kandekar@sjsu.edu}
    \and
    \IEEEauthorblockN{Saptarshi Sengupta}
    \IEEEauthorblockA{\textit{Dept. of Computer Science} \\
        \textit{San Jose State University}\\
        San Jose, United States \\
        saptarshi.sengupta@sjsu.edu}

}


\maketitle

\begin{abstract}
    Regular expressions are essential tools for developers working with text data.
    However, constructing regular expressions can be a tedious and error-prone task when dealing with longform string literals.
    To simplify this process for Java/Kotlin developers, we have developed a Java library called RegexRiot.
    The library offers a straightforward and intuitive way of constructing regular expressions by using a chain of
    function calls that describe the pattern to match.
    While similar projects exist for JavaScript, there is a lack of such libraries for Java/Kotlin.
    This project aims to bridge this gap and provide a solution for developers working with Java/Kotlin.
\end{abstract}

\begin{IEEEkeywords}
    Regular Expressions, Java, , JavaScript, Kotlin, Library, Pattern Matching, Parsing
\end{IEEEkeywords}

\section{Introduction}
When creating regex expressions, developers will often have a hard time with keeping track of exactly
what parts of the regex expression accomplish which tasks. For example, keeping track of number ranges,
such as the years 1899 up to 1991, inclusive, can become complicated as the range will include several or-blocks that
will then cause the regex expression to fill the line and possibly go on longer than what fits on the screen,
or cause a long word wrapping which can bloat the regex.
In both cases, if the programmer leaves the code as is, then comes back to it later to modify it for some reason,
there is a problem where the developer will likely have to spend time rereading the regex expression to understand
it again if the documentation was not put into place at the time.

One solution to make regex building and management easier includes the 	exttt{magic-regexp} library
which simplifies regex into separate blocks that can break down the expression into separate
groups, which one can define as subgroups.
\texttt{magic-regexp} was made as an alternative to the native JavaScript object \texttt{RegExp}.
\textbf{Figure 1} is an example which demonstrates how a regex for a phone number can be made with
magic-regexp.

\vfill\eject

\begin{figure}[htbp]
    \centering
    \label{fig:magic-regexp-phonenum-regex}
    \begin{lstlisting}[language=JavaScript]
const NEW_PHONE_RE = createRegExp(
    exactly("+")
    .optionally()
    .and("1")
    .as("country")
    .optionally()
    .and(charIn(" -").optionally())
    .and(charIn("123456789")
    .and(digit.times(3)).as("area"))
    .and(charIn("123456789")
    .and(digit.times(6)).as("rest"))
    .at.lineEnd()
    .at.lineStart()
        );
    \end{lstlisting}
    \caption{\texttt{magic-regexp} Phone Number Regex}
\end{figure}

\section{Design}

The most important features of RegexRiot when conducting research and designing the library were specified as follows:

\begin{enumerate}
    \item \textbf{Chain of function calls:}\\
          The library will provide a chain of function calls that will allow developers
          to construct regular expressions by describing the pattern they want to match.
          This approach will make it easy for developers to create complex regular expressions
          without the need for extensive knowledge of the underlying syntax.

    \item \textbf{Intuitive syntax:}\\
          The library will use an intuitive syntax that will make it easy for developers
          to understand and use regular expressions. The syntax will be designed to be easy
          to read and will provide a clear description of the pattern to match.

    \item \textbf{Support for Java and Kotlin:}\\
          The library will be developed for Java developers.
          This will provide a solution for developers working with these languages who are
          looking for an easy-to-use regular expression library.

          \vfill\eject

    \item \textbf{Wide range of functionalities:}\\
          The library will provide a wide range of functionalities for pattern matching,
          including capturing groups, non-capturing groups, character classes, quantifiers,
          and more. This will provide developers with a powerful tool for working with text data.
\end{enumerate}

With those aforementioned features taken into account,
below are some UML diagrams that depict the overall architecture of RegexRiot.

\begin{figure}[htbp]
    \centering
    \includegraphics[width=0.8\linewidth]{RegexRiot_Class_Diagram.png}
    \caption{Incomplete Class Diagram}
    \label{fig:classdiagram}
\end{figure}

\begin{figure}[htbp]
    \centering
    \includegraphics[width=0.8\linewidth]{RegexRiot_Object_Diagram.png}
    \caption{Object Diagram}
    \label{fig:objdiagram}
\end{figure}

RegexRiot is heavily inspired by the \texttt{magix-regexp} library, and as such,
follows both the imperative and declarative programming paradigms.
RegexRiot follows the imperative programming in that it is based around calling functions
to build individual components of the expression, which can then be used to construct an overall group that is a larger chunk of the master expression. With respect to the declarative paradigm, RegexRiot follows in the steps of 	exttt{magic-regexp} in that it provides a simpler and more readable syntax for the creation and maintenance of regex.

\section{Implementation}
The implementation of RegexRiot heavily depends on the usage of tokens to denote specific
primitives such as characters, integers, floats, etc. This was an important coding decision
in the context of the project because it allowed for the simplification of regex expressions
while also reducing the need for multiple comments. With the use of tokens, it is possible to
understand individual blocks of the regex expression, where the programmer would likely only
need to comment on subgroups and the whole overall group to clarify what 'chunks' accomplish
what tasks in parsing.
\\
In the following example code, RiotTokens is an interface that outlines the tokens that can
be used in conjunction with the RiotString.

\begin{figure}[htbp]
    \centering
    \label{fig:regexriot-riot-tokens}
    \begin{lstlisting}[language=Java]
public interface RiotTokens {
    /**
    * This should be safe to use as 
    * argument to RiotSet.chars()
    */
    RiotString DIGIT = 
        newLazyRiot("\\d", true),
        DOT = newLazyRiot("\\.", true),
        WORD_CHAR = newLazyRiot(
            "\\w", true),
        OPEN_BRACE = newLazyRiot(
            "\\(", true),
        CLOSE_BRACE = newLazyRiot(
            "\\)", true),
        OPEN_SQ_BRACE = newLazyRiot(
            "\\[", true),
        CLOSE_SQ_BRACE = newLazyRiot(
            "\\]", true),
        QUESTION_MARK = newLazyRiot(
            "\\?", true),
        BOUNDARY = newLazyRiot(
            "\\b", true),
        SPACE = newLazyRiot("\\s", true),
        SPACES = oneOrMore(SPACE),
        PLUS = newLazyRiot("\\+", true);
    /**
    * This is not safe to use
    * as arguments to RiotSet.chars()
    */
    RiotString ANY_CHAR = 
        newLazyRiot("."),
        LINE_START = newLazyRiot(
            "^", true),
        Line_END = newLazyRiot("$", true);

    RiotSet HEX = inSetOf(DIGIT)
        .andChars('a').through('f')
        .andChars('A').through('F'),
        HEX_LOWER = inSetOf(DIGIT)
            .andChars('a').through('f'),
        HEX_UPPER = inSetOf(DIGIT)
            .andChars('A').through('F');
    int UNLIMITED = -1;
}
    \end{lstlisting}
    \caption{\texttt{RegexRiot} RiotTokens Class}
\end{figure}

\subsection{Abbreviations and Acronyms}\label{AA}
Define abbreviations and acronyms the first time they are used in the text,
even after they have been defined in the abstract. Abbreviations such as
IEEE, SI, MKS, CGS, ac, dc, and rms do not have to be defined. Do not use
abbreviations in the title or heads unless they are unavoidable.

\subsection{Units}
\begin{itemize}
    \item Use either SI (MKS) or CGS as primary units. (SI units are encouraged.) English units may be used as secondary units (in parentheses). An exception would be the use of English units as identifiers in trade, such as ``3.5-inch disk drive''.
    \item Avoid combining SI and CGS units, such as current in amperes and magnetic field in oersteds. This often leads to confusion because equations do not balance dimensionally. If you must use mixed units, clearly state the units for each quantity that you use in an equation.
    \item Do not mix complete spellings and abbreviations of units: ``Wb/m\textsuperscript{2}'' or ``webers per square meter'', not ``webers/m\textsuperscript{2}''. Spell out units when they appear in text: ``. . . a few henries'', not ``. . . a few H''.
    \item Use a zero before decimal points: ``0.25'', not ``.25''. Use ``cm\textsuperscript{3}'', not ``cc''.)
\end{itemize}

\subsection{Equations}
Number equations consecutively. To make your
equations more compact, you may use the solidus (~/~), the exp function, or
appropriate exponents. Italicize Roman symbols for quantities and variables,
but not Greek symbols. Use a long dash rather than a hyphen for a minus
sign. Punctuate equations with commas or periods when they are part of a
sentence, as in:
\begin{equation}
    a+b=\gamma\label{eq}
\end{equation}

Be sure that the
symbols in your equation have been defined before or immediately following
the equation. Use ``\eqref{eq}'', not ``Eq.~\eqref{eq}'' or ``equation \eqref{eq}'', except at
the beginning of a sentence: ``Equation \eqref{eq} is . . .''

\subsection{\LaTeX-Specific Advice}

Please use ``soft'' (e.g., \verb|\eqref{Eq}|) cross references instead
of ``hard'' references (e.g., \verb|(1)|). That will make it possible
to combine sections, add equations, or change the order of figures or
citations without having to go through the file line by line.

Please don't use the \verb|{eqnarray}| equation environment. Use
\verb|{align}| or \verb|{IEEEeqnarray}| instead. The \verb|{eqnarray}|
environment leaves unsightly spaces around relation symbols.

Please note that the \verb|{subequations}| environment in {\LaTeX}
will increment the main equation counter even when there are no
equation numbers displayed. If you forget that, you might write an
article in which the equation numbers skip from (17) to (20), causing
the copy editors to wonder if you've discovered a new method of
counting.

    {\BibTeX} does not work by magic. It doesn't get the bibliographic
data from thin air but from .bib files. If you use {\BibTeX} to produce a
bibliography you must send the .bib files.

    {\LaTeX} can't read your mind. If you assign the same label to a
subsubsection and a table, you might find that Table I has been cross
referenced as Table IV-B3.

{\LaTeX} does not have precognitive abilities. If you put a
\verb|\label| command before the command that updates the counter it's
supposed to be using, the label will pick up the last counter to be
cross referenced instead. In particular, a \verb|\label| command
should not go before the caption of a figure or a table.

Do not use \verb|\nonumber| inside the \verb|{array}| environment. It
will not stop equation numbers inside \verb|{array}| (there won't be
any anyway) and it might stop a wanted equation number in the
surrounding equation.

\subsection{Some Common Mistakes}\label{SCM}
\begin{itemize}
    \item The word ``data'' is plural, not singular.
    \item The subscript for the permeability of vacuum $\mu_{0}$, and other common scientific constants, is zero with subscript formatting, not a lowercase letter ``o''.
    \item In American English, commas, semicolons, periods, question and exclamation marks are located within quotation marks only when a complete thought or name is cited, such as a title or full quotation. When quotation marks are used, instead of a bold or italic typeface, to highlight a word or phrase, punctuation should appear outside of the quotation marks. A parenthetical phrase or statement at the end of a sentence is punctuated outside of the closing parenthesis (like this). (A parenthetical sentence is punctuated within the parentheses.)
    \item A graph within a graph is an ``inset'', not an ``insert''. The word alternatively is preferred to the word ``alternately'' (unless you really mean something that alternates).
    \item Do not use the word ``essentially'' to mean ``approximately'' or ``effectively''.
    \item In your paper title, if the words ``that uses'' can accurately replace the word ``using'', capitalize the ``u''; if not, keep using lower-cased.
    \item Be aware of the different meanings of the homophones ``affect'' and ``effect'', ``complement'' and ``compliment'', ``discreet'' and ``discrete'', ``principal'' and ``principle''.
    \item Do not confuse ``imply'' and ``infer''.
    \item The prefix ``non'' is not a word; it should be joined to the word it modifies, usually without a hyphen.
    \item There is no period after the ``et'' in the Latin abbreviation ``et al.''.
    \item The abbreviation ``i.e.'' means ``that is'', and the abbreviation ``e.g.'' means ``for example''.
\end{itemize}
An excellent style manual for science writers is \cite{b7}.

\subsection{Authors and Affiliations}
\textbf{The class file is designed for, but not limited to, six authors.} A
minimum of one author is required for all conference articles. Author names
should be listed starting from left to right and then moving down to the
next line. This is the author sequence that will be used in future citations
and by indexing services. Names should not be listed in columns nor group by
affiliation. Please keep your affiliations as succinct as possible (for
example, do not differentiate among departments of the same organization).

\subsection{Identify the Headings}
Headings, or heads, are organizational devices that guide the reader through
your paper. There are two types: component heads and text heads.

Component heads identify the different components of your paper and are not
topically subordinate to each other. Examples include Acknowledgments and
References and, for these, the correct style to use is ``Heading 5''. Use
``figure caption'' for your Figure captions, and ``table head'' for your
table title. Run-in heads, such as ``Abstract'', will require you to apply a
style (in this case, italic) in addition to the style provided by the drop
down menu to differentiate the head from the text.

Text heads organize the topics on a relational, hierarchical basis. For
example, the paper title is the primary text head because all subsequent
material relates and elaborates on this one topic. If there are two or more
sub-topics, the next level head (uppercase Roman numerals) should be used
and, conversely, if there are not at least two sub-topics, then no subheads
should be introduced.

\subsection{Figures and Tables}
\paragraph{Positioning Figures and Tables} Place figures and tables at the top and
bottom of columns. Avoid placing them in the middle of columns. Large
figures and tables may span across both columns. Figure captions should be
below the figures; table heads should appear above the tables. Insert
figures and tables after they are cited in the text. Use the abbreviation
``Fig.~\ref{fig}'', even at the beginning of a sentence.

\begin{table}[htbp]
    \caption{Table Type Styles}
    \begin{center}
        \begin{tabular}{|c|c|c|c|}
            \hline
            \textbf{Table} & \multicolumn{3}{|c|}{\textbf{Table Column Head}}                                                         \\
            \cline{2-4}
            \textbf{Head}  & \textbf{\textit{Table column subhead}}           & \textbf{\textit{Subhead}} & \textbf{\textit{Subhead}} \\
            \hline
            copy           & More table copy$^{\mathrm{a}}$                   &                           &                           \\
            \hline
            \multicolumn{4}{l}{$^{\mathrm{a}}$Sample of a Table footnote.}
        \end{tabular}
        \label{tab1}
    \end{center}
\end{table}

\begin{figure}[htbp]
    \centerline{\includegraphics{fig1.png}}
    \caption{Example of a figure caption.}
    \label{fig}
\end{figure}

Figure Labels: Use 8 point Times New Roman for Figure labels. Use words
rather than symbols or abbreviations when writing Figure axis labels to
avoid confusing the reader. As an example, write the quantity
``Magnetization'', or ``Magnetization, M'', not just ``M''. If including
units in the label, present them within parentheses. Do not label axes only
with units. In the example, write ``Magnetization (A/m)'' or ``Magnetization
\{A[m(1)]\}'', not just ``A/m''. Do not label axes with a ratio of
quantities and units. For example, write ``Temperature (K)'', not
``Temperature/K''.

\section*{Acknowledgment}

The preferred spelling of the word ``acknowledgment'' in America is without
an ``e'' after the ``g''. Avoid the stilted expression ``one of us (R. B.
G.) thanks $\ldots$''. Instead, try ``R. B. G. thanks$\ldots$''. Put sponsor
acknowledgments in the unnumbered footnote on the first page.

\section*{References}

Please number citations consecutively within brackets \cite{b1}. The
sentence punctuation follows the bracket \cite{b2}. Refer simply to the reference
number, as in \cite{b3}---do not use ``Ref. \cite{b3}'' or ``reference \cite{b3}'' except at
the beginning of a sentence: ``Reference \cite{b3} was the first $\ldots$''

Number footnotes separately in superscripts. Place the actual footnote at
the bottom of the column in which it was cited. Do not put footnotes in the
abstract or reference list. Use letters for table footnotes.

Unless there are six authors or more give all authors' names; do not use
``et al.''. Papers that have not been published, even if they have been
submitted for publication, should be cited as ``unpublished'' \cite{b4}. Papers
that have been accepted for publication should be cited as ``in press'' \cite{b5}.
Capitalize only the first word in a paper title, except for proper nouns and
element symbols.

For papers published in translation journals, please give the English
citation first, followed by the original foreign-language citation \cite{b6}.

\begin{thebibliography}{00}
    \bibitem{b1} G. Eason, B. Noble, and I. N. Sneddon, ``On certain integrals of Lipschitz-Hankel type involving products of Bessel functions,'' Phil. Trans. Roy. Soc. London, vol. A247, pp. 529--551, April 1955.
    \bibitem{b2} J. Clerk Maxwell, A Treatise on Electricity and Magnetism, 3rd ed., vol. 2. Oxford: Clarendon, 1892, pp.68--73.
    \bibitem{b3} I. S. Jacobs and C. P. Bean, ``Fine particles, thin films and exchange anisotropy,'' in Magnetism, vol. III, G. T. Rado and H. Suhl, Eds. New York: Academic, 1963, pp. 271--350.
    \bibitem{b4} K. Elissa, ``Title of paper if known,'' unpublished.
    \bibitem{b5} R. Nicole, ``Title of paper with only first word capitalized,'' J. Name Stand. Abbrev., in press.
    \bibitem{b6} Y. Yorozu, M. Hirano, K. Oka, and Y. Tagawa, ``Electron spectroscopy studies on magneto-optical media and plastic substrate interface,'' IEEE Transl. J. Magn. Japan, vol. 2, pp. 740--741, August 1987 [Digests 9th Annual Conf. Magnetics Japan, p. 301, 1982].
    \bibitem{b7} M. Young, The Technical Writer's Handbook. Mill Valley, CA: University Science, 1989.
\end{thebibliography}

\end{document}